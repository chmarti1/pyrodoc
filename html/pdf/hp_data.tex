PYro data files are in a \verb|json| file format, which is particularly useful for flexibly representing data structures with plain text.  They are loaded as dictionaries using Python's \verb|json| package for dealing with these files.  The keys of those dictionaries can vary however the data author desires.  These files are given a \verb|*.hpd| extension short for ``PYro Data.''

As described in section \ref{sec:load}, the \verb|load()| function is responsible for seeking out and loading all data files, but for the actual process of parsing and checking data files, \verb|load()| calls the \verb|load_file()| function in the \verb|utility| module.  The utility function, \verb|load_file()|, returns a dictionary that results from parsing the data file, and throws errors or warnings if the file does not meet certain requirements.

The \verb|load()| function passes the dictionary returned by \verb|load_file()| to the appropriate class initializer (more on that later), which is trusted to do whatever it needs with the file's data.

The following sections establish more detail on how data files are structured:
\begin{itemize}
\item How are the data files formatted? 
\item What data elements does PYro require to function?
\item How are the built-in files structured
\item How can users construct their own files from the command line?
\end{itemize}



\section{Data File Format}
PYro data files use a \verb|*.hpd| extension and are written in pain text using JSON (java script object notation), for which python has support.  Detailed information about the JSON format is available at \href{json.org}{json.org}, and as usual, \href{https://docs.python.org/2/library/json.html}{Python's documentation library} has an excellent page on how the JSON package is implemented.

Each \verb|*.hpd| file contains data for a single substance.  When a file in JSON format is interpreted, it results in a dictionary of data elements that the file defines.  While the elements created by the file are entirely up to the file's author, section \ref{sec:data:min} describes the elements that PYro requires in order to make sense of the data.  The rest of the data are whatever the data's class requires to do its job.

This structure is intentionally abstract.  Very few assumptions are made about authors' intent when writing these data and their corresponding classes to allow them as much flexibility as possible.





\section{Minimal Data Structure}\label{sec:data:min}
As one might expect, there are certain requirements for PYro to be able to work with the data.  In order to be loaded successfully by \verb|load_file()|, parsing a \verb|*.hpd| file must result in a dictionary with keys \verb|id|, \verb|doc|, and \verb|class|.  Most classes will have many more, but these are the essential few for interacting with the PYro package.

\subsection{\texttt{id}}
This is the species identifier string.  This is how PYro will reference the data; for example when using the \verb|get()| or \verb|info()| functions.  For diatomic oxygen, the \verb|id| value is 'O2'.

\subsection{\texttt{doc}}
This is a documentation string.  It is expected to hold information about this particular species.  While it can be empty, it is typically a good idea to cite any sources from which the species data are taken, and this is an ideal place to include any special copyright information.  This value is used by the \verb|info()| function to describe the species.

\subsection{\texttt{class}}
This is the class identifier string.  The string contained in this field is used to look up the class for which the data is intended in the PYro registry (see section \ref{sec:reg}).  To populate the \verb|data| dictionary, \verb|load()| executes commands equivalent to
\begin{verbatim}
class_str = data_dictionary[ 'class' ]
id_str = data_dictionary[ 'id' ]
dataclass = pyro.reg.registry[ class_str ]
pyro.dat.data[ id_str ] = dataclass( data_dictionary )
\end{verbatim}

This provides a glimpse of how PYro interacts with the data classes; their initializers accept the data dictionary as their only argument.

\subsection{\texttt{fromfile}}
The \verb|fromfile| entry is the only piece of data that is directly manipulated by the PYro package.  Regardless of whether it is defined in the file, the \verb|load()| function forces this parameter to the absolute path file name of the data file from which the data was loaded.  This little bit of information can be essential for untangling redundant file problems.






\section{\texttt{igfit} Data Structure}
The ideal gas fit class uses polynomial curve fits to construct a relationship for specific heat as a function of temperature.  From those coefficients, all other properties are derived.  These particular curve fits are divided into two piecewise segments representing polynomials at high and low temperatures.

\begin{align}
c_p(T) = \left\{ \begin{array}{lr} 
\sum_{k=0}^N C_{k,1} T^k & : T<T_{split}\vspace{.5cm}\\
\sum_{k=0}^N C_{k,2} T^k & : T \ge T_{split}\\
\end{array}
\right.
\end{align}

From this principal piece of information, all other properties are derived using ideal gas relationships.  In order to implement this approach, the following data elements are used:

\subsection{\texttt{C1} and \texttt{C2}}
These lists of floating point numbers represent the coefficients of the low- and high-temperature segments.  They are arranged such that the index for each list element corresponds to the power of temperature for which it is the coefficient.  Or, in code:
\begin{verbatim}
>>> T = 400
>>> cp = 0
>>> for k in range(len(C1)):
...     cp += C1[k] * (T**k)
...
\end{verbatim}
It should be noted that this is \emph{not} actually the way the polynomial evaluation is implemented in the class.


\subsection{\texttt{h1} and \texttt{h2}}
These floating point constants serve as the integration constants when evaluating enthalpy.  They are related to, but not the same as the enthalpy of formation.  When using the lower temperature curve,
\begin{align}
h(T) &= \int c_p(T) \mathrm{d} T \nonumber\\
     &= h_1 + \sum_{k=0}^N \frac{C_k}{k+1} T^{k+1}
\end{align}


\subsection{\texttt{mw}}
This is the scalar molecular weight of the species.  It is primarily used to calculate the ideal gas constant,
\begin{align}
R = \frac{\overline{R}}{mw}.
\end{align}


\subsection{\texttt{s1}}
These floating point constants serve as the integration constants when evaluating entropy.  Entropy is obtained from the maxwell relation,
\begin{align}
\mathrm{d}s = \left(\frac{c_p}{T}\right)_p \mathrm{d}T - \left(\frac{R T}{p}\right)_T\mathrm{d}p
\end{align}

The entropy at standard pressure is obtained by integrating across temperature.
\begin{align}
s^\circ(T) &= \int \frac{c_p(T)}{T} \mathrm{d} T\nonumber\\
       &= s_1 + C_0 \ln(T) + \sum_{k=1}^N \frac{C_k}{k} T^k
\end{align}

The entropy at all pressures can be constructed by then integrating in pressure.
\begin{align}
s(T,p) = s^\circ(T) - R T \ln\left(\frac{p}{p^\circ}\right)
\end{align}
The reference pressure used is 101.3kPa.

\subsection{\texttt{Tmin}, \texttt{Tsplit}, and \texttt{Tmax}}
These parameters establish the limits on the validity of the curve fit.  \verb|Tmin| and \verb|Tmax| are the minimum and maximum temperatures for which the curve fit is expected to be accurate.  \verb|Tsplit| indicates at when to switch between the two sets of coefficients.  Below \verb|Tsplit|, \verb|C1| is used.

\subsection{\texttt{P\_ref}}
This is the pressure at which the ideal gas are taken.  For most properties, ideal gases are insensitive to pressure, but for entropy it is essential to know the reference pressure.  For most data, this is atmospheric pressure, 1.01325 bar.

\section{\texttt{igtab} Data Structure}
The \verb|igtab| data class relies on specific heat, enthalpy, and entropy to be explicitly tabulated as a function of \verb|T|.  Most of the data elements are lists of values that are interpreted as columns of an ideal gas table.  As a result all of the lists must be the same length.

The values are interpolated cubically in the interior of the data, and linearly at the extremes.  

\subsection{\texttt{cp}}
This data element is a list of specific heats corresponding to the temperatures found in the \verb|T| data element.  Since the gases belonging to this class are assumed ideal, specific heat has no pressure dependency.

\subsection{\texttt{h}}
This data element is a list of enthalpies corresponding to the temperatures found in the \verb|T| data element.  Since the gases belonging to this class are assumed ideal, enthalpy has no pressure dependency.

\subsection{\texttt{mw}}
This is the scalar molecular weight of the species.  It is primarily used to calculate the ideal gas constant.

\subsection{\texttt{s}}
This data element is a list of entropies corresponding to the temperatures found in the \verb|T| data element.  These entropies are interpreted as the entropy of the substance at reference pressure.  The actual entropy is calculated by
\begin{align}
s(T,p) = s^\circ(T) - R T \ln\left(\frac{p}{p^\circ}\right)
\end{align}
where $s^\circ(T)$ is the tabulated value.  The reference pressure is specified in the \verb|P_ref| parameter.

\subsection{\texttt{T}}
This is a list of all the temperatures used in the ideal gas table.  The values in the list must be monotonically increasing, and the length of all lists used in the table must be identical.

\subsection{\texttt{P\_ref}}
This is the pressure at which the ideal gas are taken.  For most properties, ideal gases are insensitive to pressure, but for entropy it is essential to know the reference pressure.  For most data, this is atmospheric pressure, 1.01325 bar.



\section{\texttt{mixture} Data Structure}
Rather than including explicit thermodynamic data, a mixture only need be specified by its constituents.  As a result, these data are extremely simple to represent.

\subsection{\texttt{contents}}
The \verb|contents| dictionary is a mapping between the species present in the mixture and their respective quantities.  The keys are the \verb|id| strings of each species present.  If the \verb|bymass| data element is True, then the floating point values indicate the respective masses present in the mixture.  Otherwise, the values indicate a mole or volume quantity present.  These numbers need not be normalized into percentages.

\subsection{\texttt{bymass}}
This is a boolean constant indicating whether the entries of the \verb|contents| should be interpreted as mases or molar quantities.





\section{Managing Data Files from the Command Line}
Because PYro is designed for users to manipulate and add data, there are utilities that are intended to help with the process.  For example, users do not need to manually edit \verb|*.hpd| files.  For most purposes, it will be easier and more reliable to use the command line.

The \verb|dat| module's \verb|new()| and \verb|updatefiles()| functions are incredibly powerful for creating and modifying data.  Users can create a data dictionary from a script or command line, and the \verb|new()| function will import it into the data dictionary as if it had resulted from a \verb|*.hpd| file; calling the necessary class initializer.

For example, the H35 mixture data entry was created using code quite similar to the following:
\begin{verbatim}
>>> import pyro
>>> H35 = {'id':'H35', 'class':'mixture'}
>>> H35['doc'] = '35% H2 by volume and balance Ar.'
>>> H35['bymass'] = False
>>> H35['contents'] = {'H2':35., 'Ar':65.}
>>> pyro.new(H35)
\end{verbatim}

Once created, these changes to the data dictionary would only be temporary were it not for the \verb|updatefiles()| funciton.  This utility compares the data dictionary with the files for changes in either the files or the dictionary.  Any discrepancies are reported and the user is guided through an interactive session to bring the two into agreement.

New files can be created, data entries that were removed since load can be either deleted or suppressed.  Suppressed files have a leading '~' in their file name to direct the \verb|load()| function to disregard them.  Redundant data definitions found in the files can also be resolved by deleting or suppressing files.

\begin{verbatim}
>>> hp.dat.updatefiles()
\end{verbatim}

Of course, users can determine the state of the data dictionary themselves before changes are actually implemented.  The command
\begin{verbatim}
>>> CHK = hp.dat.load(verbose=True, check=True)
\end{verbatim}
prints a summary of all \verb|*.hpd| files discovered (including suppressed ones) and their comparison with the data in memory.  The CHK dictionary keys include detailed data for the comparison: lists of species identifiers \verb|added|, \verb|changed|, or \verb|suppressed|; a \verb|suppressed| dictionary mapping species id strings to the redundant files defining them; and the \verb|data| dictionary that would have resulted had the load operation been run normally.
