Most of the functions that users will ever need in order to interact with PYro are exposed in the base module.  Everything else is contained in four subordinate modules; \verb|utility|, a collection of miscellaneous functions, \verb|reg|, a registry of all data classes used in the PYro package, and \verb|dat|, which manages raw data and associates them with their respective classes.


\section{PYro Base Module}
The PYro base module exposes all of the functionality most users will ever need.  Though the package has a number of measures to make dynamically incorporating new data and configuration options easy, these are all contained in subordinate modules.

The base module exposes two functions; \verb|info()| and \verb|get()|, which are responsible for retrieving information about the data available and retrieving the data classes themselves.  It also exposes a dictionary, \verb|config|, which contains all of the user configurable parameters that modify PYro's behavior.

When the PYro package is imported, it searches for and constructs a registry of classes that know how to interpret data.  Then, it searches for data defining specific and defines a dictionary of objects representing them.  This all occurs automatically within the \verb|reg| and \verb|dat| modules, which are discussed in more detail in sections \ref{sec:reg} and \ref{sec:dat}.

\subsection{\texttt{config}}
PYro's behavior depends on a number of configurable parameters that are contained in the \verb|config| dictionary.  The values of these parameters are loaded when the package is imported using the \verb|load_config()| function in the \verb|utility| module.  Some of the parameters are user configurable (like file and directory locations) and others are merely intended for reference (like the package version).  Details on these parameters are provided in chapter \ref{sec:config}.

\subsection{\texttt{info()}}
The \verb|info()| function retrieves information on the various species currently loaded into memory.  When it is evoked without any arguments, it prints a table to \verb|stdout| listing the names, date modified, and class type of the species in memory.  When called with the string name of a substance as an argument, \verb|info()| prints detailed information on the species data.

\subsection{\texttt{get()}}
The \verb|get()| function accepts the string name of a species to be retrieved from memory.  If such a species exists in memory, \verb|get()| returns its class object, which in turn exposes methods for calculating properties.

\section{\texttt{reg}: Registry Module}\label{sec:reg}
PYro is designed to dynamically incorporate new data and classes either released as updates or written by users.  The registry contained in the \verb|reg| module is responsible for managing all of the data classes.

Since it is impossible to anticipate all of the possible data structures or substances users might want to incorporate into the PYro system, it is essential that the methods used for calculating properties not be built in to the package.  Instead, the \verb|reg| module dynamically loads class definitions at load.

The \verb|reg| module exposes a dictionary, \verb|registry|, a class \verb|__baseclass__|, and a function, \verb|regload()|.  

\subsection{\texttt{registry}}
The \verb|registry| dictionary is a mapping between a data type string, and the corresponding class definition for handling that data type.  For example, a typical PYro installation will give the following output:
\begin{verbatim}
>>> pyro.reg.registry['igfit']
<class 'pyro.reg.igfit'>
\end{verbatim}
The data type strings are necessarily the same as the class name.

When data are loaded in the \verb|dat| module, they use this dictionary to find their corresponding class definitions.  Exactly how this is accomplished is described in detail in section \ref{sec:dat}.


\subsection{\texttt{regload()}}
The \verb|regload()| function is responsible for populating the \verb|registry| dictionary.  It is called automatically when the \verb|reg| module is first imported, but developers may wish to call it from the command line to incorporate changes.

The entires of the \verb|registry_dir| configuration parameter constitute a list of all locations where registry files are supposed to be found.  \verb|regload()| checks the contents of each directory, in the order listed, for *.py files without a leading underscore or period (`\_' or `.').  Any children of the \verb|__basedata__| class created in these files are added to the \verb|registry| dictionary, and all other objects are ignored.

By default, the only registry directory is \verb|pyro/registry|, but system administrators may want to allow users to include their own registry directories.  Great care should be taken, however, to prevent standard users from accessing registry directories that will be used by other users, as this creates a security risk.  Chapter \ref{sec:config} explains more on this.

\subsection{\texttt{\_\_basedata\_\_}}
This is a prototype class for all PYro data classes.  In order for the \verb|reg| module to recognize a data class, \verb|issubclass( thisclass, __basedata__)| must return true.  To help developers write their own classes, \verb|__basedata__| has detailed documentation, there is an \verb|_example.py| file in the \verb|pyro/registry| directory, and the entire process is described in more detail in chapter \ref{sec:classes}.



\section{\texttt{dat}: Data Module}\label{sec:dat}
The \verb|dat| module is responsible for loading, retrieving, checking, and manipulating the thermodynamic data of the various classes.  For this task, the data module exposes the \verb|data| dictionary, which maps a species name to an object for manipulating it.

\subsection{\texttt{data}}
Each species loaded into PYro resides in the \verb|data| dictionary, so that the following lines are equivalent:
\begin{verbatim}
>>> pyro.get('CO2')
>>> pyro.dat.data['CO2']
\end{verbatim}

The data dictionary is populated automatically when the \verb|dat| module is imported and can be manually reloaded using the \verb|load()| function.  For more details on loading, checking, and saving changes to the data, see the \verb|load()|, \verb|updatefiles()|, and \verb|new()| functions documentation.

\subsection{\texttt{clear()}}
This function empties the \verb|data| dictionary, and can be useful for developers.

\subsection{\texttt{load()}}
The \verb|load()| function is extremely important to PYro for standard users and developers alike.  It is executed automatically when the \verb|dat| module is loaded.  When called without arguments, it scans all directories listed in the \verb|data_dir| parameter in the \verb|config| dictionary for files with a ``*.hpd'' extension and no leading underscore or decimal (`\_' or `.').  Files that meet these criteria are passed to the \verb|load_file()| function in the \verb|utility| module which, if all goes well, returns a dictionary containing the class data.  Finally, the dictionary is passed to the appropriate class initializer in the \verb|reg| module, and the resulting object is added to the \verb|data| dictionary.

For developers and users interested in adding their own data files, the \verb|load()| function can do quite a bit more.  If called with the ``check'' directive set to True, rather than loading into the \verb|data| dictionary, it performs a mock load operation and compares the results with the data currently in memory.  It checks for data entries that have been newly created, deleted, or altered since load.  It also lists any files that exhibit redundant definitions for the same entry, and lists any *.hpd files that were suppressed by adding a leading period or underscore to their file names.

The intent is for developers to construct or edit data at the command line and use the built in tools to incorporate them into the PYro system.

\subsection{\texttt{updatefiles()}}
When called with no arguments, \verb|updatefiles()| initiates an interactive process for bringing the current data dictionary in agreement with the files in the various data directories.  The \verb|load()| function is called with the check parameter set to list all discrepancies.

Files with redundant entry definitions and files for entries that have been removed can be suppressed or deleted.  Edits to the data in memory can be written to the appropriate files, and files for new data entries can be created automatically.

\subsection{\texttt{new()}}
Useful for users who want to build scripts for creating new data, the \verb|new()| function creates a new data entry from a data dictionary as if it had been loaded from a file.  When used in conjunction with the \verb|updatefiles()| funciton, new data sets can be easy to implement without ever needing a detailed understanding of how PYro stores and retrieves data.

\section{\texttt{utility}: Miscellanea Module}
This module is a collection of functions, imported modules, and classes that need never be exposed to the user.  This includes error types, functions for generating error and warning messages, and most importantly, it includes functions for interacting with data files.  While these objects are documented, few users or developers should really need to interact with them.


