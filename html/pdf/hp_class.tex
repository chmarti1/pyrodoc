Through the class registry module described in section \ref{sec:reg}, users are afforded substantial freedom in writing their own PYro data classes.  Provided that authors obey a few guidelines, the PYro system will automatically detect new class definition files and handle the rest.

Authors can create new data class definitions by adding \verb|*.py| files to any of the directories listed in the \verb|reg_dir| configuration entry (see section \ref{sec:config}).  Each file with a \verb|*.py| extension and without a leading \verb|'_'| or \verb|'.'| is executed.  Any class definitions built on the \verb|__basedata__| class (exposed in the \verb|reg| module) are added to the \verb|registry|.

These files are executed in a context where the \verb|pyro| package is available as a global for access to any other modules (e.g. \verb|pyro.utility.np| for access to NumPy).  Similarly, the \verb|__basedata__| is a global.  Authors who need access to additional modules or want to create other global references may do so without concern since only objects that have \verb|__basedata__| as a parent are imported into the registry.

\section{\texttt{\_\_basedata\_\_}}
The best introduction to the anatomy of a PYro class is probably through the \verb|__basedata__| class itself.  Authors who write their own classes are encouraged to define them as children of the \verb|__basedata__| class, since it has prototypes for all the important functions of a PYro class, and exposes some useful tools (especially \verb|_vectorize()|).

\subsection{\texttt{\_\_init\_\_()}}
The responsibilities of a PYro data class begin with the \verb|__init__()| function.  It accepts the class's data dictionary as its only argument, which is then copied into the object's \verb|data| member.  It completes its responsibilities by executing the \verb|__basetest__()| and then the \verb|__test__()| functions.

Authors writing their own \verb|__init__()| functions for classes should probably evoke the \verb|__basetest__()|'s \verb|__init__()| function with the line
\begin{verbatim}
def __init__(self,data):
    ... special code ...
    super(myclass, self).__init__(args)
\end{verbatim}
This is an easy way to include all the error checking code already included in \verb|__basetest__()|.

Alternately, authors may wish to explicitly include code from the \verb|__init__()| function
\begin{verbatim}
def __init__(self,data):
    ... special code ...
    self.data = data.copy()
    self.__doc__ = self.data['doc']
    self.__basetest__()
    self.__test__()
\end{verbatim}
Authors who use this approach may want to be aware that their classes will not track with updates to the \verb|__basetest__| class.

\subsection{\texttt{\_\_basetest\_\_()} and \texttt{\_\_test\_\_()}}
The \verb|__basetest__()| function raises an error if any of the names listed in the class's \verb|mandatory| list member are not found in the data dictionary.  The \verb|mandatory| list is initialized to include \verb|'class'|, \verb|'doc'|, and \verb|'id'|, but any other data elements are strictly class specific.  Authors should be careful to edit their class's \verb|mandatory| list to include every data element on which their code relies to operate so any missing dependencies can be caught and reported when the data is first loaded.  A line in the \verb|__init__()| function can do the trick.
\begin{verbatim}
mandatory += ['new','data','elements']
\end{verbatim}

Additional checks on the robustness of the data or other class integrity checks can be implemented in the \verb|__test__()| function, which is intended to be user defined.  By default, it does nothing.  For example, authors might want to raise an error from inside \verb|__test__()| if some part of the data does not meet expectations.

\subsection{\texttt{\_vectorize()}}
The \verb|_vectorize()| function is made available to authors to help with the common task of dealing with property arguments of different dimensions.  The built-in classess build all of their calculations on NumPy arrays, and use the \verb|_vectorize()| function to automatically enforce that temperature and pressure inputs are of compatible dimensions.  See the function docstring for more information.

\section{Authors' Responsibilities}
In order to produce a functional data class, users will need to do the following:
\begin{enumerate}
\item Create a class definition that depends on the \verb|__basedata__| class in a file.
\item Place the containing file in a directory found in the \verb|reg_dir| search path.
\item Compose the class's definition of the \verb|mandatory| list to include each data element required for the class to function properly.
\item Assemble data files that PYro can load (see section \ref{sec:data})
\item Compose class methods that can access the resulting data dicitonary to calculate properties.
\end{enumerate}

If authors want the class to be compatible with the built-in mixture class, they should be sure to include the following member methods:
\begin{itemize}
\item \verb|cp()| calculates constant-pressure specific heat in kJ/kg/K
\item \verb|cv()| calculates constant-volume specific heat in kJ/kg/K
\item \verb|d()| calculates the density in kg/m$^3$
\item \verb|e()| calculates the internal energy in kJ/kg
\item \verb|h()| calculates the enthalpy in kJ/kg
\item \verb|k()| calculates the specific heat ratio
\item \verb|mw()| calculates the molecular weight in kg/kmol
\item \verb|s()| calculates the entropy in kJ/kg/K
\end{itemize}
Units should be kJ for energy, seconds for time, kg for mass, m for length/volume, and kmoles for molecular counts.

