\chapter{Introduction}

\PM\ is a Python-based package for calculating the thermodynamic properties of fluids.  That includes liquids, gases, and plasmas.  It is written in pure Python, and the core algorithms only depend on the Numpy package for back-end numerical and array support.  This document is intended to serve as a reference manual for interpreting, using, or even writing your own \PM\ data sets.  

Probably, no reader will need to digest this handbook in its entirity.  The goal is to include all the important information in a single portable document.  Chatper \ref{ch:start} on Getting Started is probably the one that most users will need, since it describes installation and basic use of the software.

For users who want a review of the thermodynamic properties, this introduction goes into detail on the fundamental definitions for the properties used.  First, we talk briefly about the ideas of \emph{primary} and \emph{basic properties}, which \PM\ uses to organize its work on the back-end.  Then, we will briefly define each property with a modest attempt to describe its origins and relevance.

For the curious or for developers who may want to add their own substance models, the inner workings of the package are described in some detail in Chapters \ref{ch:regdat}, \ref{ch:ig}, \ref{ch:mp}, and \ref{ch:num}.

%
% The Properties
%
\section{The Properties}

Here, we begin with a brief review of the thermodynamic properties of liquids and gases that \PM\ calculates.  For a detailed development of these properties, the reader should consult an introductory text on thermodynamics\footnote{For example, Cengel and Boles, \emph{Thermodynamics}, McGraw Hill.  Any edition will do.}.  

In the author's opinion, contemporary undergraduate texts on thermodynamics do the reader a disservice by ignoring the deeply intuitive (and terribly important) underpinnings provided by the kinetic theory of gases.  Kinetic theory and thermodynamics have a complicated relationship because they were developed in parallel (often in tragic contention with one another), but the limits imposed by many of the common assumptions (such as ideal or perfect gas) cannot be deeply understood without kinetic theory.  To the interested reader, the author would recommend Jeans's 1949 introductory text\footnote{James Jeans, \emph{Introduction to the Kinetic Theory of Gases}, Cambridge University Press, 1949.}.  It is old, but it is brief, accessible, inexpensive, and approaches the subject only requiring that the reader have a grasp of calculus, geometry, and introductory mechanics.  

\PM\ is primarily concerned with the \emph{thermodynamic} properties listed in Table \ref{tab:properties}.  There is no single set of units used to express these properties since \PM\ uses a user-configurable unit system.  The units systems given in Table \ref{tab:properties} are itemized in Table \ref{tab:units}.  All of Chapter \ref{ch:units} is devoted to describing \PM's treatment of units.

\begin{table}
\centering
\caption{Thermodynamic properties, their symbols, and their unit systems}\label{tab:properties}
\begin{tabular}{|cccl|}
\hline
Symbol & In-Code & Units & Description\\
\hline
$T$ & \verb|T| & T & Temperature\\
$p$ & \verb|p| & P & Pressure\\
$\rho$ & \verb|d| & M / V & Density\\
$x$ & \verb|x| & - & Quality\\
\hline
$R$ & \verb|R| & E / M / T & Ideal gas constant\\
$W$ & \verb|mw| & Ma / Mo & Molecular weight\\
\hline
$a$ & \verb|a| & L / t & Speed of sound\\
$e$ & \verb|e| & E / M & Internal energy\\
$f$ & \verb|e| & E / M & Free (Helmholtz) energy\\
$g$ & \verb|e| & E / M & Gibbs energy\\
$h$ & \verb|h| & E / M & Enthalpy\\
$s$ & \verb|s| & E / M / T & Entropy\\
\hline
$c_p$ & \verb|cp| & E / M / T & Constant-pressure specific heat\\
$c_v$ & \verb|cv| & E / M / T & Constant-volume specific heat\\
$\gamma$ & \verb|gam| & - & Specific heat ratio\\
\hline
\end{tabular}
\end{table}

\begin{table}
\centering
\caption{Classes of units, their entry in \texttt{pm.config}, and their defaults.}\label{tab:units}
\begin{tabular}{|cccl|}
\hline
Unit & \verb|config| entry & Default & Description\\
\hline
E & \verb|unit_energy| & kJ & Energy\\
L & \verb|unit_length| & m & Length\\
M & \verb|unit_matter| & kg & Matter (molar or mass)\\
Ma & \verb|unit_mass| & kg & Mass\\
Mo & \verb|unit_molar| & kmol & Molar\\
P & \verb|unit_pressure| & bar & Pressure\\
t & \verb|unit_time| & s & Time\\
T & \verb|unit_temperature| & K & Temperature\\
V & \verb|unit_volume| & m$^3$ & Volume\\
\hline
\end{tabular}
\end{table}

In theory, thermodynamics can make equal use of almost any pair of properties to discover the others, and as of version 2.2.0, \PM\ is quite flexible about which two are used.  While this kind of convenience is part of the purpose of a high-level software tool, users may notice that some combinations are faster than others, some produce more numerical inaccuracies than others, and some are simply not possible.

\subsection{Basic Properties}\label{sec:intro:basic}

Users will see better performance when the state is specified with \emph{basic properties}:
\begin{itemize}
\item temperature, $T$,
\item pressure, $p$,
\item density, $\rho$,
\item specific volume, $v$,
\item quality, $x$.
\end{itemize}
Except for quality, the state of any substance may always be specified with any two basic properties, and a basic property may be paired with any non-basic property.

Other properties, like 
\begin{itemize}
\item internal energy, $e$,
\item enthalpy, $h$, 
\item entropy, $s$,
\end{itemize}
can also be used to specify the thermodynamic state, but the user should be aware of their limitations.  Specifying them forces \PM\ to use a slower back-end algorithm with inherent numerical precision limitations.  There may also be some limitation on how these properties are combined.

Tables \ref{tab:props:ig} and \ref{tab:props:mp} show which property combinations are allowed to be combined to define the thermodynamic state.  Combinations are marked with ``X'' when they would either be self-contradictory or redundant.  Every combination along the diagonal is obviously redundant, but simultaneously specifying specific volume and density is also redundant.  Specifying enthalpy, internal energy, and temperature of an ideal gas is also redundant, so these combinations are ``invalid.''

In Table \ref{tab:props:mp}, quality, $x$, is listed as being invalid in combination with any property that is not temperature or pressure, but the reason is less obvious than in the ideal gas cases.  For example, it is not at all redundant to specify an enthalpy and a quality.  However, for many qualities, there are multiple states that give the same enthalpy, so, for example, \texttt{T(h,x)}, is not theoretically even a function.  In more practical terms, the algorithm would converge to inconsistent results, and may even diverge due to singularities in the iteration.

Meanwhile, combinations of enthalpy, entropy, and internal energy are not supported purely for pragmatic reason.  The multi-dimensional iteration that results is easy to implement for specific cases, but it presents significant stability challenges in the general case.  There are plans to include this functionality in the future, but only when codes can be shown to deliver garanteed convergence over the entire domain.

\begin{table}
\centering
\caption{Property combinations that are supported by ideal gas classes in version 2.4.5.  \checkmark = supported, X = not valid, o = not supported.}\label{tab:props:ig}
\begin{tabular}{|c|cccc|ccc|}
\hline
  & \texttt{T} & \texttt{p} & \texttt{d} & \texttt{v} & \texttt{h} & \texttt{e} & \texttt{s}\\
\hline  %  & T   p   d   v   h   e   s
\texttt{T} & X & \checkmark & \checkmark & \checkmark & X & X & \checkmark\\
\texttt{p} & \checkmark & X & \checkmark & \checkmark & \checkmark & \checkmark & \checkmark\\
\texttt{d} & \checkmark & \checkmark & X & X & \checkmark & \checkmark & \checkmark\\
\texttt{v} & \checkmark & \checkmark & X & X & \checkmark & \checkmark & \checkmark\\
\hline
\texttt{h} & X & \checkmark & \checkmark & \checkmark & X & X & \checkmark\\
\texttt{e} & X & \checkmark & \checkmark & \checkmark & X & X & \checkmark\\
\texttt{s} & \checkmark & \checkmark & \checkmark & \checkmark & \checkmark & \checkmark & X\\
\hline
\end{tabular}
\end{table}

\begin{table}
\centering
\caption{Property combinations that are supported by the multi-phase class in version 2.4.5.  \checkmark = supported, X = not valid, o = not supported.}\label{tab:props:mp}
\begin{tabular}{|c|ccccc|ccc|}
\hline
  & \texttt{T} & \texttt{p} & \texttt{d} & \texttt{v} & \texttt{x} & \texttt{h} & \texttt{e} & \texttt{s}\\
\hline  %  & T   p   d   v   h   e   s
\texttt{T} & X & \checkmark & \checkmark & \checkmark & \checkmark & \checkmark & \checkmark & \checkmark\\
\texttt{p} & \checkmark & X & \checkmark & \checkmark & \checkmark & \checkmark & \checkmark & \checkmark\\
\texttt{d} & \checkmark & \checkmark & X & X & X & \checkmark & \checkmark & \checkmark\\
\texttt{v} & \checkmark & \checkmark & X & X & X & \checkmark & \checkmark & \checkmark\\
\texttt{x} & \checkmark & \checkmark & X & X & X & X & X & X \\
\hline
\texttt{h} & \checkmark & \checkmark & \checkmark & \checkmark & X & X & o & o\\
\texttt{e} & \checkmark & \checkmark & \checkmark & \checkmark & X & o & X & o\\
\texttt{s} & \checkmark & \checkmark & \checkmark & \checkmark & X & o & o & X\\
\hline
\end{tabular}
\end{table}

\subsection{Primary Properties}\label{sec:intro:primary}

Every thermodynamic model supported by \PM\ is formulated to calculate the substance's many properties in terms of temperature and one other property.  For ideal gases, it is temperature and pressure, and for multi-phase substances it is temperature and density.  The properties that are used in the back-end to calculate the others are called the \emph{primary properties} for each model.  \PM\ is designed so that users do not need to be aware of this distinction unless they are concerned with speed or extreme numerical precision.

For example, an ideal gas model allows entropy (see sec. \ref{sec:ig:hs}) to be calculated from a polynomial in terms of temperature and pressure.  While this theoretically provides a means for calculating temperature from entropy and pressure, $T(s,p)$, the entropy polynomial is far too complex to be inverted explicitly, and defining a separate polynomial for every combination of properties is impractical.  That means that calculating temperature from entropy and pressure is theoretically sound, but it requires a (relatively) computationally expensive numerical iteration routine - $s(T,p)$ is evaluated repeatedly until a value of $T$ can be found that produces a value ``close enough'' to the desired value of $s$.  Chapter \ref{ch:num} discusses this process in some detail.

Each class is designed around its own primary properties, and the performance will always be best if users can write their code to use these properties when they are available.  Table \ref{tab:primary} lists the current \PM\ substance classes and their primary properties.

For most problems of engineering relevance, temperature and pressure $(T,p)$ are the favorite since both are readily measured, and both are of immediate importance to fluid machinery design.  Fortunately, all of the common properties of ideal gases can be conveniently constructed directly from these two properties.  Unfortunately, that is not the case with real fluids.

It is intuitive that when gases are compressed into tighter spaces, the distance between the molecules becomes a vitally important parameter for predicting the substance's properties.  Pressure is not a convenient metric of that distance, but density is.  For this reason, nearly all non-ideal gas properties are modeled in terms of $(T,\rho)$, and pressure has to be calculated indirectly. 

\begin{table}
\centering
\caption{The \PM\ substance classes and their primary property pairs}\label{tab:primary}
\begin{tabular}{|lcc|}
\hline
Desription & Class & Prim.\\
\hline
Shomate Eqn. & \verb|ig| & $(T,p)$\\
NASA poly. & \verb|ig2| & $(T,p)$\\
IG mixture & \verb|igmix| & $(T,p)$\\
Span \& Wagner & \verb|mp1| & $(T,\rho)$\\
\hline
\end{tabular}
\end{table}

\subsection{Density, $\rho$}

It is convenient to begin our discussion of properties with density since it is the easiest to define.

Density is the quantity of a substance per unit volume it occupies in space.  It can be described as a number of molecules or mass per unit space.  When described with molar units, it is usually called concentration, but \PM\ does not make that distinction.

When the unit volume is very tiny, this is a poorly defined quantity.  For example, as the volume shrinks to be about the same size as the distance between molecules, the density one might measure would vary hugely as individual molecules entered and left the region of space.  However, as the volume grows within a thermodynamically homogeneous region, the ratio of matter to volume converges to a consistent well-defined value.

This introduces the idea that the study of thermodynamics is essentially a careful study of averages over a large population of mechanical bodies.  It is important that we study a quantity of a substance large enough that the quasi-random motions of individual molecules do not weigh heavily in our measurements.  On the other hand, our measurements must consider a region of space small enough that the properties do not vary significantly.  We may consider a region to be thermodynamically homogeneous if any two of its halves exhibit identical thermodynamic properties.

The density of the gas in molecules per unit volume is
\begin{align}
n = \frac{N}{V}.
\end{align}
where $V$ is the volume of the region, and $N$ is the number of molecules.  These numbers are extremely large, and early in the history of chemistry, there was no way to know the true number of molecules in a sample anyway.  So, the use of molar quantities was of great utility.

The density in molar units is a slight variation on $n$,
\begin{align}
\overline{\rho} = \frac{N}{N_a V} = \frac{\overline{N}}{V},
\end{align}
where $N_a$ is Avagadro's number, and $\overline{N}$ is the number of moles in the sample.  As we will see in Section \ref{sec:units:molar}, other molar units exist, but the same formula applies.  In this way, the density of molecules may be expressed either in a literal count per unit volume, $n$, or in a number of moles per unit volume, $\overline{\rho}$.

In mass units, the density is merely multiplied by the molecular weight of the species,
\begin{align}
\rho &= m_0 \frac{N}{V}\\
 &= \frac{\overline{N} W}{V} = \frac{N W}{N_a V}\nonumber
\end{align}
when $m_0$ is the mass of a single molecule, and $W$ is the more commonly used molecular weight in atomic mass units.

Since $\rho$ is not available in the ASCII character set, \PM\ uses \verb|d| to represent density.

\subsection{Specific volume, $v$}

Specific volume is the volume occupied by a unit of matter, and is calculated as the inverse of density.
\begin{align}
v \equiv \frac{1}{\rho}.
\end{align}
Density is the property with the clearer fundamental definition.  After all, how much space can be said to be occupied by a gas molecule?  Formulations that address the average distance between molecules essentially reduce to defining specific volume as the inverse of density.

However, specific volume is mathematically identical to other intensive properties because it expresses the quantity of interest (volume) per unit matter.  That makes it extremely convenient for a number of thermodynamic calculations.

\subsection{Temperature, $T$}

Temperature scales were originally developed as quantitative means for describing hot and cold, but they were developed before we had more physical descriptions for their meaning.  After all, the existence of molecules and atoms was still being hotly debated while temperature scales were already in wide scientific and engineering use.

We now understand temperature to be an observable measure of the molecular translational kinetic energy of a substance.  In a gas, the molecules are free to translate through space, and temperature is proportional to their kinetic energy.  In a liquid or solid, the same energy manifests as molecular vibration within the confines of the intermolecular forces.

For a gas,
\begin{align}
\langle \frac{1}{2} m u^2 \rangle = \frac{3}{2} k T,
\end{align}
Here, $m$ is the mass of an individual molecule, $\langle u^2 \rangle$ is the mean square of velocity, $k$ is the Boltzmann constant, and $T$ is the temperature in absolute units.

The ITS-90 temperature scale establishes an international standard for the definition of temperature in terms of the triple points of various pure substances.  Many of the property models included in \PM\ were formulated when the previous ITPS-68 was the international temperature scale, but the changes were so minute that the uncertainties in the properties usually dominate\cite[p.10]{janaf:1:1998}.  As a result, they are treated as interchangeable for the purposes of this handbook.

See chapter \ref{sec:units:temperature} for more information on the formal definition of temperature.

\subsection{Pressure, $p$}\label{sec:intro:p}

Pressure is the static force exerted by a fluid on a surface.  It is quantified in force per unit area of the surface, and it always acts normal to the surface facing into the surface (away from the fluid).  

In a gas, pressure is due to the impact of molecules on the surface.  Pressure may be increased by increasing either their velocity (temperature) or their quantity (density).  Because these effects are linear in a gas, this leads to the famous ideal gas relationship between density, temperature, and pressure.  

In a liquid or solid, intermolecular forces that cause pressure are persistent instead of momentary (due to collisions in a gas).  Under these conditions, even slight changes in intermolecular spacing causes huge changes in pressure, making the substance quite stiff in comparison to gases.  In this case as well, increasing temperature makes molecules vibrate more violently in their equilibrium with each other, so at a consistent average density, the force applied to a neighboring surface will increase.  This is why substances appear to expand as they are heated.

\subsection{Internal Energy, $e$}\label{sec:intro:e}

Energy can be stored in a fluid in many ways.  In gases, for example, the molecules translate with great speed (see temperature), the molecules vibrate and rotate, and there is incredible energy stored in the chemical bonds of molecules.  However, all of these are dwarfed by the energy contained in the forces binding the nucleus of each atom.

We account for \emph{all} of these energies simultaneously with the internal energy, $e$, which has units energy per matter (e.g. J/kg).  It is neither practical nor necessary to tally all of these energies in an absolute fashion.  Especially since most applications will have no release of nuclear energy, it is practical, instead, to describe how the substance's energy changes relative to some reference state.  This is not unlike defining a reference height for gravitational potential energy calculations in classical mechanics.  The choice is arbitrary and has no bearing on the result, but it can drastically simplify the calculations.

Therefore, we say that the internal energy, $e$ is the sum of vibrational, rotational, translational, chemical, and nuclear energies contained in a thermodynamically homogeneous unit matter, subtracted by the same sum at some reference state.

Some readers will be scandalized that \PM\ uses $e$ in favor of the traditional $u$ for internal energy.  This is a genuine (if futile) attempt to untangle the web of contradictory variable use between thermodynamics and fluid mechanics.  $u$ is reserved for velocity, $v$ for volume, and $e$ for energy.

\subsection{Enthalpy, $h$}\label{sec:intro:h}

When a fluid of any kind is flowing, it caries its internal energy with it, but it also does mechanical work as it flows.  The mechanical work done by a moving fluid is $p \d V$, where $p$ is the pressure exerted and $\d V$ is a differential volume displaced by the fluid.  If we were to imagine that the volume were displaced while the fluid is expanding or contracting, the same idea applies to a fluid whether it is flowing or not.  Per unit mass, this can be expressed as $p \d v$ (when $v$ is specific volume).  Integrated over an isobaric (constant-pressure) process, this becomes simply $p v$.

It is a matter of convenience for engineers and physicists that deal with fluid power, that we define enthalpy as the sum of internal energy and fluid power,
\begin{align}
h \equiv e + pv = e + \frac{p}{\rho}.\label{eqn:intro:h}
\end{align}
Of course, most processes aren't \emph{actually} isobaric, so extra terms will tend to appear in energy balances (like $\d h - v \d p$).

Enthalpy is most commonly used in the analysis of any flowing fluid such as in heat exchangers, combustors and burners, chemical reactors, compressors, turbines, valves, etc...  Even though internal energy might be argued to be more ``fundamental,'' enthalpy is usually tabulated instead because of its widespread use.

Since internal energy and enthalpy are related, their reference states \emph{must} be consistent with one another.  By tradition in both the NIST-JANAF and NASA data sets, the enthalpy of a substance is defined at 298.15 K and 1 bar, and values at other conditions are calculated from specific heat and the physics of ideal gases.  Since the \PM\ multi-phase models permit only pure substances, their reference states do not need to be interchangeable.  By tradition, the internal energy and entropy of these substances are defined as zero for the liquid at the triple point.

See the ideal gas (Chapter \ref{ch:ig}) and multi-phase (Chapter \ref{ch:mp}) chapters for more information on how the reference states are handled.

\subsection{Entropy, $s$}\label{sec:intro:s}

The idea of entropy is born with the Clausius statement of the Second Law of Thermodynamics, which says that a reversible cyclic series of processes that include heat transfer must obey
\begin{align}
\oint \frac{\delta q}{T} \equals_{rev.} 0,
\end{align}
where $T$ is the temperature of the substance and $\delta q$ is the addition of heat.  It is important to emphasize that this is only true of processes that result in a continuous cycle where the fluid ends at the same thermodynamic state from which it began (like in an engine or refrigeration cycle).

The first law would be satisfied by any cycle where the work and heat transfer merely summed to zero, but the Clausius equality implies something deeper.  Heat is not a property of the substance, but heat added reversibly in ratio with the temperature consistently returns to zero when the substance returns to its original thermodynamic state.  That implies the existence of a new (and terribly important) property.  Clausius termed that property entropy,
\begin{align}
\d s \defas_\mathrm{rev.} \frac{\delta q}{T}\label{eqn:intro:s}
\end{align}

From the first law of thermodynamics, $\delta q = \d e + p \d v$, and we have a way to calculate entropy from the other properties
\begin{align}
T \d s &\equals_{rev.} \d e + p\d v\label{eqn:intro:ds1}\\
 &= \d h - v \d p\label{eqn:intro:ds2}
\end{align}
This calculation is performed by putting the substance through a hypothetical reversible process in which there is neither chemical reaction, nor field work (e.g. gravitational or electrical).

Just like with temperature, there is a parallel (mathematically consistent) definition of entropy from statistical mechanics.  The above definition of entropy had long been in use, when Boltzmann found that Entropy can also be calculated from the probability of the substance obtaining a thermodynamically equivalent state.  The probability of gas molecules spontaneously exhibiting a specific set of positions and velocities is astronomically low, but there is an immense quantity of dissimilar positions and velocities that would give the gas precisely the same temperature, internal energy, pressure, density, and other properties.  For a given volume populated with a given substance with a given number of molecules with a given amount of total energy, how numerous are those states in comparison with all the other possible states?

Entropy is high when the number of equivalent states is high.  Entropy is low when the number of equivalent states is low.  The idea was fiercely resisted when Boltzmann and his contemporaries first argued it.  From one point of view, it is astonishing that the idea was first discovered by brilliant insight \emph{without} statistics.

Just like internal energy, it is theoretically possible to tally all of the possible states and calculate the probability of each, but that task is neither practical nor necessary.  Instead, it is convenient to define the entropy as zero at some reference state and then deal merely in changes in entropy as defined by (\ref{eqn:intro:s}).  In the case of ideal gases, the entropy is known to be precisely zero at absolute zero temperature, but some multi-phase models (like water) set entropy (and internal energy) of the liquid phase to zero at the triple point.

\subsection{Free (Helmholtz) energy, $f$}\label{sec:intro:f}

The free energy is written in terms of internal energy and entropy,
\begin{align}
f \defas e - Ts \label{eqn:intro:f}.
\end{align}
While free energy has long been used as a tool for calculating the mechanical work available from a substance at a given temperature, modern equation-of-state substance models use $f$ as the property from which all others are calculated.

There is tremendous potential for confusion, because various authors have also referred to Gibbs energy as ``free energy.''  Here, we strictly reserve that phrase to refer to $e - Ts$.

Sources can be found using $a$, $\phi$, $\alpha$, and many other characters to represent free energy.  We adopt $f$ for its relationship to the word ``free,'' and to reserve these variables for other uses.


\subsection{Gibbs energy, $g$}\label{sec:intro:g}

The Gibbs energy is defined as,
\begin{align}
g &\defas e + \frac{p}{\rho} - Ts\label{eqn:intro:g}\\
 &=  h - Ts\nonumber\\
 &= f + \frac{p}{\rho}\nonumber.
\end{align}
It has historically been called \emph{free enthalpy} or Gibbs free energy as well.  Here, we adopt the recommendations of the IUPAC and refer to it simply as \emph{Gibbs energy}.

Its use is primarily in systems with chemical or phase reactions.  In these applications, properties are written in terms of the \emph{chemical potential}, $\mu_i$, which is the contribution of each constituent to the internal energy,
\begin{align*}
\d e &= T \d s - p \d v + \sum_i \mu_i \d y_i\\
\d \overline{e} &= T \d \overline{s} - p \d \overline{v} + \sum_i \overline{\mu}_i \d \chi_i.
\end{align*}
It bears mentioning that these potentials can be related to the enthalpies of formation and reference entropies of the pure constituents.

Here, chemical reactions or phase transitions cause changes in the mass fractions, $y_i$, or mole fractions, $\chi_i$, which can consume or release energy.  So, for a hypothetical reaction with constant entropy and volume,
\begin{align}
\mu_i = \left(\frac{\partial e}{\partial y_i}\right)_{s,v}.
\end{align}
However, specifying properties in terms of volume and entropy in a reacting experiment is not particularly practical.  Examining chemical potential in terms of enthalpy merely transposes the expression into constant entropy and pressure.

However, when we examine the differential of $g$,
\begin{align*}
\d g &= (\d e) + (p\d v + v \d p) - (T \d s + s \d T),
\end{align*}
and substitute to find the chemical potentials,
\begin{align*}
\d g - (p \d v + v \d p) + (T \d s + s \d T) &= T \d s - p \d v + \sum_i \mu_i \d y_i\\
\d g &= v \d p - s \d T + \sum_i \mu_i \d y_i.
\end{align*}
Temperature and pressure are among the most commonly specified properties when performing thermodynamic calculations, so it is extremely convenient to write the chemical potential in terms of Gibbs energy,
\begin{align}
\mu_i = \left(\frac{\partial g}{\partial y_i}\right)_{T,p}.
\end{align}
In fact, in the special case of ideal gas mixtures, $g = \sum y_i g_i(T,p_i)$, so $\mu_i = g_i(T,p_i)$.

\subsection{Specific Heats, $c_p$ $c_v$}\label{sec:intro:c}

Of the properties so far defined, only temperature, pressure, and density can be directly measured.  The specific heats are vitally important to the study of a substance, because they can be conveniently directly measured and the other properties calculated from them.

Specific heat, $c$, is the amount of heat per mass of a substance, $\delta q$, required to obtain a small increase in temperature, $\delta T$,
\begin{align}
c \equiv \frac{\delta q}{\delta T}.
\end{align}

Especially when dealing with gases, we have to be more specific because substances have a tendency to expand when they are heated.  One measures a different value for specific heat depending on whether or not expansion is allowed. 

For any process, energy will be conserved, so
\begin{align}
\delta q = \d e + p\d v.\label{eqn:intro:1stlaw}
\end{align}
Here, $\delta q$ is heat added per mass of the substance, $\d e$ is the change in internal energy, $p$ is the pressure, and $\d v$ is the change in volume per unit mass.

When heat is applied in such a manner that the substance's volume is constant, $\d v = 0$, the mechanical work is zero, and all of the heat is stored as internal energy.
\begin{align}
\delta q |_v &= \d e\nonumber\\
 &= \left(\frac{\partial e}{\partial T}\right)_v \d T
\end{align}
Thus, the constant-volume specific heat is, by definition, 
\begin{align}
c_v \equiv \left(\frac{\delta q}{\delta T}\right)_v = \left(\frac{\partial e}{\partial T}\right)_v.\label{eqn:intro:cv}
\end{align}
In this process, the substance's pressure will rise (sometimes sharply) as it is heated in a rigid container.

When one considers addition of heat under constant pressure, a trick application of the chain rule for the term, $pv$, lets us transition the differential on volume into a differential on pressure.  The definition for enthalpy (\ref{eqn:intro:h}) appears naturally.
\begin{align}
\delta q &= \d e + p \d v + v\d p - v \d p\nonumber\\
 &= \d(e + pv) - v\d p
\end{align}

When heat is added while pressure is constant (like in an atmospheric gas), $\d p=0$, and 
\begin{align}
\delta q |_p &= \d h\nonumber\\
 &= \left( \frac{\partial h}{\partial T}\right)_p \d T
\end{align}
Thus, constant-pressure specific heat is, by definition,
\begin{align}
c_p &\equiv \left(\frac{\delta q}{\delta T}\right)_p = \left( \frac{\partial h}{\partial T}\right)_p.\label{eqn:intro:cp}
\end{align}


\subsection{Speed of sound, $a$}\label{sec:intro:a}

Like specific heats, the speed of sound is the property of a substance that can be measured in the laboratory to high precision, making it a valuable tool.  Pressure wave propagation in a substance obeys the equation,
\begin{align}
\frac{\partial^2 p}{\partial t^2} + \left( \frac{\partial p}{\partial \rho} \right)_s \frac{\partial^2 p}{\partial x^2} = 0.\label{eqn:intro:wave}
\end{align}
A wave with wavenumber, $k$, propagating at constant speed, $a$, might take the form
\begin{align*}
p = \sin\Big(k (x + a t)\Big).
\end{align*}
This satisfies (\ref{eqn:intro:wave}) only when
\begin{align}
a^2 \defas \left( \frac{\partial p}{\partial \rho} \right)_s.\label{eqn:intro:sound}
\end{align}
So, wave speed in a substance is the square root of the isentropic rise in pressure with increasing density.  

Wave speed is usually denoted with the variable, $c$, which we reserve for specific heats.  Authors can be found using $w$ and $a$, and we have selected the latter.  

%
% 
%
