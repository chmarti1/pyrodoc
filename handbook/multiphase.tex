\chapter{Multi-phase substance models}\label{ch:mp}

Unlike ideal gas properties, it is sensible to formulate multi-phase properties in terms of temperature and density instead of temperature and pressure.  Modeling phase changes with temperature and pressure as independent variables requires a discontinuity at the phase change.  On the other hand, using density requires no such discontinuity, and it even permits modeling meta-stable states.

Historically, there have been countless approaches to the problem, beginning with the so-called ``viral'' terms inspired by Van der Waals forces in early work on the kinetic theory of gases.  This approach began an evolution of increasingly complex variants of the ideal gas law, but parallel models for the specific heat are needed to establish a total substance model.  Eventually, this approach fell out of popularity in favor of single property models from which all other properties could be derived.  

If these general models have some basis in physics, these approaches have often been argued to be superior because they make it possible to derive properties that are not so easily measured.  For example, the specific heat or the speed of sound may be quite practical to measure precisely in a laboratory, and a substance model that reliably predicts them throughout the thermodynamic domain may also provide good predictions for a quantity that is harder to measure, like the precise location of the critical point.

It is possible that future releases of \PM\ may include other formulations, but the wide availability of works providing coefficients for more contemporary methods has made it practical to use a single multi-phase class to model pure substances.  There are works extending these models to mixtures like air, but those are not yet implemented.

\section{General formulation for \texttt{mp1}}\label{sec:mp:mp1}

The models supported by the \texttt{mp1} class have explicit formulations for the Holmholtz free energy (or simply free energy) in terms of temperature and density, and all other properties are calculated from that formulation.  Free energy is defined as
\begin{align}
a(T,\rho) &\equiv e(T,\rho) - T s(T,\rho)\\
 &= h - \frac{p}{\rho} - T s.
\end{align}
By constructing the formulation from a bank of terms inspired by theoretical formulae for the intermolecular forces, there is greater hope of reducing the number of terms needed. 

The formulation that we describe in this chapter is sometimes referred to as a ``Span and Wagner'' fit for Helmholtz free energy.  Though some version of the approach had already been applied successfully to specific substances over a decade earlier \cite{stewart:1991, setzmann:1991}, the need for a form that a code could apply generically to many substances prompted Span and Wagner's three-part 2003 paper defending its broad use for thermodynamic properties of fluids \cite{span:2003:1, span:2003:2, span:2003:3}.  An expanding body of papers by Span, Wagner, Lemon, Jacobsen, and others a library of formulations for substances that use a standard bank of terms to construct the free energy equation.

In its independent implementation of this model, \PM\ is a modest open-source Pythonic alternative to REFPROP \cite{refprop}, the NIST code widely adopted for implementing these general forms.

\subsection{Nondimensionalization}

It is generally sound practice to nondimensionalize formulae in all but the most trivial numerical problems.  Ensuring that parameters vary on the order of unity helps reduce the severity of numerical errors, and if the nondimensionalization is performed with special attention to the underlying physics, then it is highly likely that the complexity of the formulae required will be reduced.

All of the substance models in the first multi-phase class use
\begin{subequations}
\begin{align}
\alpha(\tau,\delta) &= \frac{a(T,\rho)}{R T}\\
\tau &= \frac{T_c}{T}\\
\delta &= \frac{\rho}{\rho_c}.
\end{align}
\end{subequations}

The free energy is normalized by the quantity $R T$, which is motivated by (\ref{eqn:ig:efromt}) from the study of ideal gases.  Normalizing temperature and density by the critical point values acknowledges that the critical point is a natural scale for the important phenomena in this substance.  The choice to make $\tau$ scale like the inverse of temperature is motivated by the Boltzmann and Maxwell distributions for molecular velocity, in which temperature appears in a denominator.

The formulation for dimensionless free energy, $\alpha$, is split into two parts: the ideal gas part, $\alpha^o$, and the residual part, $\alpha^r$.  So, the total free energy is
\begin{align}
\alpha(\tau, \delta) = \alpha^o(\tau, \delta) + \alpha^r(\tau, \delta)
\end{align}
This approach separates the problem of needing to model the energy contained in molecular vibration and translation of a large number of independent oscillators ($\alpha^o$) from the problem of modeling the effects of intermolecular forces with increasing density ($\alpha^r$).

\subsection{Ideal gas portion of free energy}
The ideal gas portion of the free energy can be constructed from a specific heat model, just as other properties were for the ideal gas substances in Chapter \ref{ch:ig}.  The ideal gas enthalpy is merely the integral of specific heat, and ideal gas entropy can be similarly constructed in terms of specific heat, temperature, and density,
\begin{subequations}
\begin{align}
h &= h_0 + \int_{T_0}^T c_p \d T\\
s &= s_0 + \int_{T_0}^T \frac{c_p}{T} \d T - R \ln\left(\frac{\rho}{\rho_0}\right) - R\ln\left(\frac{T}{T_0}\right).
\end{align}
\end{subequations}
When these are transposed into the dimensionless parameters, $\tau$ and $\delta$,
\begin{align*}
h &= h_0 - T_c \int_{\tau_0}^\tau \frac{c_p}{\tau^2} \d \tau\\
s &= s_0 - \int_{\tau_0}^\tau \frac{c_p}{\tau} \d \tau - R \ln\left(\frac{\delta \tau_0}{\delta_0 \tau}\right).
\end{align*}
When these are used to calculate the dimensionless free energy,
\begin{align}
\alpha^o &= \frac{h - RT - Ts}{RT} = \frac{h \tau}{R T_c} - 1 - \frac{s}{R}\nonumber\\
 &= \frac{h_0}{RT_c} \tau - \frac{s_0}{R} - 1 + \ln\left(\frac{\delta \tau_0}{\delta_0 \tau}\right) + \left(-\tau\int_{\tau_0}^\tau \frac{c_p}{R \tau^2} \d \tau + \int_{\tau_0}^\tau \frac{c_p}{R \tau} \d \tau \right)\label{eqn:mp1:aodef}
\end{align}
Note that, for ease of interpretation, we have adopted a lazy notation with $\tau$ both inside and outside of the integrals.  We will adopt a more strict notation below.

The difficulty in formulating the specific heat of even an ideal gas is introduced in Section \ref{sec:intro:e}.  It is, briefly, that portions of energy that a substance stores in translation (thermal energy) versus the many internal forms of vibration depends heavily on the temperature (and intermolecular distance for a real substance).  A fully detailed model would need to include molecular vibration, electronic, magnetic energies, and quantom mechanical effects.  Approaches to this problem range from a few relatively complicated terms derived from first principles to purely empirical approaches with many simple polynomial terms.  In this discussion, we will spend a little time addressing those approaches that are most common to the ideal gas portion of the \verb|mp1| model.

In the ideal gas models described so far in Chapter \ref{ch:ig}, generic empirical polynomials do a reasonable job of matching the phenomena that dominate there with the advantage that a couple of standard algorithms can consistently evaluate them.  The caveat is that they require piece-wise definitions to adequately capture all of the relevant features of the curve, but the \verb|mp1| model covers the entire temperature range with a single expression.  Even though the residual terms are responsible for the intermolecular forces, low-temperature gas behaviors still need to be considered in the ideal gas portion.  As a result, many high-performing published models tend to include a mix of polynomial and physics-based terms derived from quantum theory.

Models that include terms derived from quantum mechanics can compactly match the decrease in specific heat seen at low temperature.  The specific heat of many independent oscillators assuming a distribution of discrete energies adopts the form
\begin{align}
\frac{c_p(\tau)}{R} = \ldots + b m^2 \frac{\tau^2 \exp(m\tau)}{\left(\exp(m\tau) - 1\right)^2} + \ldots\nonumber,
\end{align}
where $b$ represents the magnitude of the feature, and $m$ is a dimensionless temperature (normalized by $T_c$) at which the feature occurs.

The integrals of specific heat can be collectively simplified by inverting the integration-by-parts procedure with a series of substitutions, so that
\begin{align}
-\tau \int_{\tau_0}^\tau \frac{c_p(t)}{Rt^2} \d t + \int_{\tau_0}^\tau \frac{c_p(t)}{Rt} \d t = - \int_{\tau_0}^\tau \left[ \int_{\tau_0}^t \frac{c_p(t')}{Rt'^2} \d t'\right] \d t.
\end{align}
This has the fortunate effect of canceling the $\tau^2$ in the quantum component of specific heat, making its integral straightforward.
\begin{align}
-\int_{\tau_0}^\tau &\int_{\tau_0}^t \frac{b m^2 \exp(m t')}{\left(\exp(m t') - 1\right)^2} \d t' \d t  \ldots\nonumber\\
 &= \int_{\tau_0}^\tau \left[ \frac{b m}{\exp(m t) - 1} - \frac{b m}{\exp(m \tau_0) - 1} \right] \d t \nonumber\\
 &= \int_{\tau_0}^\tau \frac{b m\exp(-mt)}{1 - \exp(-mt)}\d t - \frac{b m}{\exp(m\tau_0)-1}(\tau - \tau_0) \nonumber\\
 &= b\ln(1-\exp(-m\tau)) - b\ln(1-\exp(-m\tau_0)) + \ldots\nonumber\\
 & \hspace{2em} -\frac{b m}{\exp(m\tau_0)-1}(\tau - \tau_0)
\end{align}

The same approach could be applied to the integration of the polynomial, but it is probably more easily dealt with in separate terms.  For a sum of terms with arbitrary exponents,
\begin{align}
-\tau \int_{\tau_0}^\tau &\sum_k c_k t^{k-2} \d t + \int_{\tau_0}^\tau \sum_k c_k t^{k-1} \d t \ldots \nonumber\\
&= -\tau \left[c_1 \ln t + \sum_{k\ne 1} \frac{c_k t^{k-1}}{k-1} \right]_{\tau_0}^\tau  + \left[c_0 \ln t + \sum_{k\ne 0} \frac{c_k t^{k}}{k} \right]_{\tau_0}^\tau\nonumber\\
&= - \tau c_1 \ln \tau + \tau c_1 \ln \tau_0 - \left[\sum_{k\ne 1} \frac{c_k \tau^k}{k-1}\right] + \tau \left[\sum_{k\ne 1} \frac{c_k \tau_0^{k-1}}{k-1} \right] + \ldots\nonumber\\
&\hspace{2em}  + c_0 \ln \tau - c_0 \ln \tau_0 + \left[ \sum_{k\ne 0} \frac{c_k \tau^{k}}{k} \right] - \left[ \sum_{k\ne 0} \frac{c_k \tau_0^{k}}{k} \right].
\end{align}
Note that this analysis does not require that values of $k$ be limited to integers.

Collectively, these integrals motivate the form that is now commonplace in contemporary free-energy-based models,
\begin{subequations}
\begin{align}
\alpha^o(\tau,\delta) = \ln \delta + (c_0 - 1 + c_1\tau) \ln \tau + \alpha^o_0(\tau) + \alpha^o_1(\tau)\label{eqn:mp1:ao}.
\end{align}
In many models, the $\tau \ln \tau$ term is omitted, implying that there was no linear term on $\tau$ (proportional to $1/T$) in the formulation of specific heat.  This is often seen in models that model specific heat with a constant and a series of quantum terms with no polynomial terms.  The remaining terms are a polynomial on $\tau$, $\alpha^o_0$, and the quantum terms, $\alpha^o_1$. which are only functions of temperature.  The $\alpha^o_1$ term includes a sum of all of the quantum functions,
\begin{align}
\alpha^o_1(\tau) = \sum_j b_j \ln\left(1 - \exp( -m_j \tau ) \right)\label{eqn:mp1:ao:q},
\end{align}
\end{subequations}
and the $\alpha^o_0(\tau)$ term is merely a polynomial.  In many models, it is a constant and a linear term only, representing the terms that appear in (\ref{eqn:mp1:aodef}).

Because much of the complexity of the specific heat is captured by the $\alpha^o_1$ terms, the polynomial, $\alpha^o_0$, may only contain a constant and a linear term.  Still, exponentials and logarithms are numerically expensive, so this model is usually substantially slower than the polynomial ideal gas models.

In the data files, the ideal gas terms are contained in a dictionary called the \texttt{"AOgroup"}.  An \texttt{"AOgroup"} entry in the file might appear like the example below.  For more information on data files, see Section \ref{sec:regdat:data}.  For more information on polynomial coefficient lists, see Section \ref{sec:num:poly1}.

% This comment numbers the code columns.  There are 53 before a line overrun.
%        1         2         3         4         5
%2345678901234567890123456789012345678901234567890123
\begin{lstlisting}[language=Python]
"AOgroup" : {
    "Tscale" : <Tc>,
    "dscale" : <dc>,
    "logt" : <c0>,
    "tlogt" : <c1>,
    "coef0" : <p coefficients>,
    "coef1" : [[<m0>, <b0>], [<m1>, <b1>], <...>]
}
\end{lstlisting}

\subsection{Residual portion of free energy}
The residual (or real-fluid) portion of free energy attempts to account for everything that the ideal gas portion does not.  Many of these terms become especially important when the substance density is high, because intermolecular forces are specifically ignored in the ideal gas model.  The model is divided into three groups of terms,
\begin{subequations}
\begin{align}
\alpha^r &= \alpha^r_0(\tau, \delta) + \alpha^r_1(\tau, \delta) + \alpha^r_2(\tau, \delta)
\end{align}
Many of these terms are physically motivated, but that discussion is not included here.

The first group of terms is a series of polynomials multiplied by exponentials of powers of density,
\begin{align}
\alpha^r_0 &= p_0(\tau, \delta) + \sum_{k=1}^K \exp\left(-\delta^k\right) p_k(\tau, \delta)\\
& p_k = \sum_i c_i \tau^{a_i} \delta^{b_i}.\label{eqn:alpha:r0}
\end{align}
While it is not explicitly forbidden, the $\delta$ exponent, $b$, is neither zero nor negative in these models, so $\alpha^r_0$ terms with the exponential multipliers vanish near $\delta \rightarrow 0$ and $\delta \rightarrow \infty$.  These terms model the complicated behaviors that occur near the phase change (when $\delta$ is on the order 1).

In most models, the first group ($\alpha^r_0$) constitutes the largest number of terms (often dozens).  The exponents in the polynomial expansion are sometimes integers, but they are often rationals (which permits efficient evaluation using the method described in Section \ref{sec:num:poly2}), and they are rarely long decimal values.  

The second group of terms, $\alpha^r_1$, is a Gaussian function multiplied by polynomial terms of $\tau$ and $\delta$.  It was first introduced in 1991 by Setzmann et al. to better model the severely distorted surface that occurs near the critical point in oxygen.  Most models only use a few of these terms at most, and they are usually centered (by $\epsilon$ and $\gamma$) very close to the critical point.  From to the 1991 paper, ``The opinion has been widely held that analytic equations of state covering the whole fluid region are not able to represent the properties in the critical region.'' The authors continue to say, ``We will show that our new equation is
able to represent the existing experimental information on the thermodynamic properties of methane in the critical
region at least as well as [competing models].'' \cite{setzmann:1991}

The third group, $\alpha^r_2$, is a Gaussian centered on exactly the critical point, but the coefficient, $\Delta$, is a scaled distance from the critical point.  The term appears in a 1994 paper modeling carbon dioxide \cite{span:1994} and again in a 2014 revision to the properties of water and steam \cite{iapws:2014}, but in conspicuously few others.  While its formulation is not explained in detail in these sources, it allows for a more complicated shape from the term without the computational expense of additional polynomial terms.  It seems to have been designed to reduce the number of Gaussian terms required to produce the desired result at the critical point.

\begin{align}
\alpha^r_1 &= \sum_{l=0}^L c_l \delta^{d_l} \tau^{t_l} \exp\left( -a_l(\delta-\epsilon_l)^2 - b_l(\tau-\gamma_l)^2 \right)\label{eqn:alpha:r1}\\
\alpha^r_2 &= \sum_{m=0}^M c_m \Delta_m{^{b_m}} \delta \exp\left(-C_m(\delta-1)^2 - D_m(\tau-1)^2 \right)\label{eqn:alpha:r2}\\
 &\Delta_m = \left[1-\tau + A_m\left((\delta-1)^2\right)^{1/2\beta_m} \right]^2 + B_m\left((\delta - 1)^2\right)^{a_m}.
\end{align}
\end{subequations}
At first glance, these formulations appear to be an inefficient tangled mess of nested exponents.  However, quantities like $\tau-1$ and $\delta-1$ may be positive or negative, and the exponents are not integers.  It is numerically elegant to square these quantities (requiring just a multiplication operation) before raising them to decimal or fractional powers.

In the data files, these terms are encoded in a format shown below.  The \texttt{coef0} member contains a list of 2D polynomial coefficients.  Each polynomial, $p_k$, corresponds to an exponential term, $\exp(-\delta^k)$.  The \texttt{coef1} and \texttt{coef2} members are nested lists with each containing the necessary coefficients and exponents to construct the terms of $\alpha^r_1$ and $\alpha^r_2$ respectively.  
% This comment numbers the code columns.  There are 53 before a line overrun.
%        1         2         3         4         5
%2345678901234567890123456789012345678901234567890123
\begin{lstlisting}[language=Python]
"ARgroup" : {
    "Tscale" : <Tc>,
    "dscale" : <dc>,
    "coef0" : [<p0 coef>, <p1 coef>, ...]
    "coef1" : [[<t>, <d>, <b>, <a>, <gamma>, <epsilon>, <c>], ...]
    "coef2" : [[<a>, <b>, <beta>, <A>, <B>, <C>, <D>, <c>], ...]
}
\end{lstlisting}

For more information on polynomial coefficients, see Section \ref{sec:num:poly2}.  It is worth emphasizing that \texttt{coef0} is a four-deep nested list.
\begin{lstlisting}[language=Python]
"coef0" : [# This begins the list of polynomials
             [# This begins the first polynomial
                 [# First sub-polynomial
                     [<a>, <b>],
                     [<alpha>, <beta>],
                     [<i0>, <j0>, <c0>],
                     <...>
                 ], <...more sub-polynomials?...>
             ],
             [# This begins the exp(-delta) term
                 <...polynomial definition...>
             ],
             [# This begins the exp(-delta**2) term
                 <...polynomial definition...>
             ],
             <...>
         ]
             
\end{lstlisting}

\subsection{Derivatives of free energy}

As we will address in the next section, it is also important to evaluate up to the second derivative of these terms.  This is sufficiently cumbersome to deserve some treatment here.  The efficient evaluation of derivatives of polynomials is addressed in Section \ref{sec:num:poly2}, so we only need to treat the other aspects of these functional forms.

The ideal gas portion in (\ref{eqn:mp1:ao}) has derivatives
\begin{subequations}
\begin{align}
\alpha^o_\tau &= (c_0-1)\tau^{-1} + c_1 (1 + \ln \tau) + \alpha^o_{0,\tau} + \alpha^o_{1,\tau}\\
\alpha^o_\delta &= \delta^{-1}\\
\alpha^o_{\tau\tau} &= -(c_0-1)\tau^{-2} + c_1 \tau^{-1} + \alpha^o_{0,\tau\tau} + \alpha^o_{1,\tau\tau}\\
\alpha^o_{\tau\delta} &= 0\\
\alpha^o_{\delta\delta} &= -\delta^{-2}
\end{align}
\end{subequations}
The polynomial, $\alpha^o_0(\tau)$, and its derivatives can be evaluated separately.  All that remains is to determine the derivatives of $\alpha^o_1(\tau)$.

\begin{subequations}
\begin{align}
\alpha^o_{1,\tau} &= \sum_j \frac{b_j m_j}{\exp(m_j \tau) - 1}\\
\alpha^o_{1,\tau\tau} &= \sum_j \frac{b_j m_j^2 \exp(m_j \tau)}{(\exp(m_j \tau) - 1)^{2}}
\end{align}
\end{subequations}

Derivatives of the residual portion in (\ref{eqn:mp1:ar}) can be calculated individually.  Much of $\alpha^r_0$ in (\ref{eqn:alpha:r0}) can be calculated trivially from the efficient derivatives of the polynomials. 


\section{Calculation of properties}\label{sec:mp1:properties}

The \texttt{mp1} class calculates all substance properties from temperature, density, $\alpha$, and its derivatives.  In this section, the formulae that are used to calculate the various properties are developed from first principles.

In this development, it will be useful to have this form of the first law:
\begin{align}
T\d s &= \d e - \frac{p}{\rho^2} \d \rho\label{eqn:mp1:law1}.
\end{align}

By definition,
\begin{align}
a &= e - Ts \label{eqn:mp1:a}.
\end{align}
Its derivatives with respect to temperature and density may be simplified with some help from (\ref{eqn:mp1:law1}),
\begin{align}
\left(\frac{\partial a}{\partial T}\right)_\rho &= \left(\frac{\partial e}{\partial T}\right)_\rho - s - T \left(\frac{\partial s}{\partial T}\right)_\rho &=& \ -s\label{eqn:mp1:at}\\
\left(\frac{\partial a}{\partial \rho}\right)_T &= \left(\frac{\partial e}{\partial \rho}\right)_T - T \left(\frac{\partial s}{\partial \rho}\right)_T &=& \ \frac{p}{\rho^2} \label{eqn:mp1:ad}.
\end{align}

The properties below are calculated in a dimensionless form.  In this way, this document is relieved from needing to be sensitive to units -- including molar versus mass. 
 
\subsection{Pressure}\label{sec:mp1:p}

The pressure (and compressibility factor) can be calculated explicitly from (\ref{eqn:mp1:ad}),
\begin{align}
\frac{p}{\rho R T} &=  \frac{\rho}{R T}\left(\frac{\partial a}{\partial \rho}\right)_T\nonumber\\
&= \delta \alpha_\delta\label{eqn:mp1:p}.
\end{align}

\subsection{Entropy}\label{sec:mp1:s}

Entropy appears explicitly in (\ref{eqn:mp1:at}), so it only needs to be transformed into terms of $\alpha$.  First, it is helpful to observe that derivatives with respect to temperature may be transposed into derivatives with respect to $\tau$ 
\begin{align}
\frac{\partial}{\partial T} = -\frac{\tau}{T} \frac{\partial}{\partial \tau}.\nonumber
\end{align}
Therefore, 
\begin{align}
s &= -\left(\frac{\partial a}{\partial T}\right)_\rho \nonumber\\
&= -\frac{\partial}{\partial T} R T \alpha\nonumber\\
&= -R \alpha + R \tau \alpha_\tau
\end{align}
So, normalized by $R$, entropy is
\begin{align}
\frac{s}{R} = \tau \alpha_\tau - \alpha\label{eqn:mp1:s}
\end{align}

\subsection{Internal energy}\label{sec:mp1:e}

Internal energy can be calculated in terms of $a$ and $s$ from (\ref{eqn:mp1:a}) and (\ref{eqn:mp1:s}),
\begin{align}
\frac{e}{R T} &= \frac{a}{RT} + \frac{s}{R} \vspace{2em} =\ \tau \alpha_\tau\label{eqn:mp1:e}.
\end{align}

\subsection{Enthalpy}\label{sec:mp1:h}

Enthalpy is trivial to construct from (\ref{eqn:mp1:p}) and (\ref{eqn:mp1:e}).  By definition, 
\begin{align}
\frac{h}{R T} &= \frac{e}{RT} + \frac{p}{\rho R T}\nonumber\\
 &= \tau \alpha_\tau + \delta \alpha_\delta.\label{eqn:mp1:h}
\end{align}

\subsection{Specific heats}\label{sec:mp1:c}

Constant-volume specific heat is obtained by differentiating internal energy by temperature.
\begin{align*}
c_v &= \frac{\partial e}{\partial T} = e_\tau \frac{\d \tau}{\d T}\\
 &= -e_\tau \frac{\tau}{T}
\end{align*}
From (\ref{eqn:mp1:e}), $e$ is simply $R T_c \, \alpha_\tau$, so
\begin{align}
\frac{c_v}{R} = - \tau^2 \alpha_{\tau\tau}\label{eqn:mp1:cv}.
\end{align}

Constant-pressure specific heat is not so simply obtained since pressure is not one of the independent variables of the formulation.  Derivations for $c_p$ usually begin with enthalpy, but it will become aparent that the calculation benefits from the prior derivation for $c_v$.  Therefore, from (\ref{eqn:intro:1stlaw}) we may write
\begin{align*}
c_p &= \left(\frac{\partial e}{\partial T}\right)_p - \frac{p}{\rho^2} \left(\frac{\partial \rho}{\partial T}\right)_p.
\end{align*}

However, because the \texttt{mp1} formulation is always in terms of temperature and density, derivatives with constant pressure must be transposed to derivatives on $T$ and $\rho$, 
\begin{align*}
\left(\frac{\partial e}{\partial T}\right)_p &=  \left(\frac{\partial e}{\partial T}\right)_\rho +  \left(\frac{\partial e}{\partial \rho}\right)_T  \left(\frac{\partial \rho}{\partial T}\right)_p.
\end{align*}
Of course, the same must be done for the derivative of density,
\begin{align*}
\left(\frac{\partial \rho}{\partial T}\right)_p &= -\frac{ \left(\frac{\partial p}{\partial T}\right)_\rho }{ \left(\frac{\partial p}{\partial \rho}\right)_T }.
\end{align*}
Substituting,
\begin{align*}
c_p &= \left(\frac{\partial e}{\partial T}\right)_\rho - \left(\left(\frac{\partial e}{\partial \rho}\right)_T - \frac{p}{\rho^2}\right) \frac{ \left(\frac{\partial p}{\partial T}\right)_\rho }{ \left(\frac{\partial p}{\partial \rho}\right)_T }.
\end{align*}
The first term (the derivative of interneral energy with respect to temperature) is merely $c_v$.  The second term may be simplified using (\ref{eqn:mp1:law1}), so
\begin{align*}
c_p &= c_v - T\left(\frac{\partial s}{\partial \rho}\right)_T \frac{ \left(\frac{\partial p}{\partial T}\right)_\rho }{ \left(\frac{\partial p}{\partial \rho}\right)_T }.
\end{align*}

Finally, this is in a form that can be conveniently substituted for free energy.  Using (\ref{eqn:mp1:at}) and (\ref{eqn:mp1:ad}),
\begin{align*}
\left(\frac{\partial s}{\partial \rho}\right)_T &= - \frac{\partial^2 a}{\partial T \partial \rho} = - \frac{R}{\rho}\left( \delta \alpha_\delta - \tau \delta \alpha_{\tau\delta} \right)\\
\left(\frac{\partial p}{\partial T}\right)_\rho &= \rho^2 \frac{\partial^2 a}{\partial T \partial \rho} = R \rho \left( \delta \alpha_\delta - \tau \delta \alpha_{\tau\delta} \right)\\
\left(\frac{\partial p}{\partial \rho}\right)_T &= 2 \rho \left(\frac{\partial a}{\partial \rho}\right)_T + \rho^2 \left(\frac{\partial^2 a}{\partial \rho^2}\right)_T \ldots\\
 &= RT \left( 2\delta \alpha_\delta + \delta^2 \alpha_{\delta\delta} \right)
\end{align*}
So, 
\begin{align}
\frac{c_p}{R} = \frac{c_v}{R} + \frac{\left(\delta \alpha_\delta - \tau \delta \alpha_{\tau\delta} \right)^2}{2\delta \alpha_\delta + \delta^2\alpha_{\delta\delta}}
\end{align}

\subsection{Liquid-vapor line}

Technically, the Maxwell conditions for phase equilibrium provide enough information for an algorithm to iteratively calculate the temperature and densities (and all other properties) from these properties.  However, the process is sufficiently numerically expensive to demand the use of additional empirical curves so the saturation line can be efficiently calculated.  This means that there are tiny numerical discrepancies between the saturation properties and the \emph{actual} saturation properties that one would predict using the Maxwell criteria.  

Formulae are provided for
\begin{itemize}
\item liquid saturation density given temperature,
\item vapor saturation density given temperature,
\item saturation pressure given temperature.
\end{itemize}
From them, all other properties can be obtained along the saturation line.  The \texttt{mp1} class exposes these properties through methods: \texttt{ds}, \texttt{ps}, \texttt{Ts} (not to be confused with \verb|T_s|), \texttt{es}, \texttt{hs}, and \texttt{ss}.  All of these depend on the three basic empirical formulae.

These formulae are calculated from dimensionless temperature, 
\begin{align}
\theta = \frac{T}{T_c},
\end{align}
which is the inverse of $\tau$.  Each of the formulae for a saturation property, $p$, take one of four types:

{\bf 0. A polynomial on $\theta$} 
\begin{align}
p(\theta) = \sum_k c_k \theta^{a_k}
\end{align}
This form is commonly used for liquid-solid saturation lines (not yet implemented in version 2.1.0).

{\bf 1. A polynomial on $1-\theta$}
\begin{align}
p(\theta) = \sum_k c_k (1-\theta)^{a_k}
\end{align}
This form is usually used for the saturation pressure.

{\bf 2. An exponential of a polynomial on $1-\theta$}
\begin{align}
p(\theta) = \exp \left( \sum_k c_k (1-\theta)^{a_k} \right)
\end{align}
This form is usually used for the liquid saturation density.

{\bf 3. An exponential of a polynomial on $1-\theta$ with a $1/\theta$ multiplier}
\begin{align}
p(\theta) = \exp \left( \frac{1}{\theta} \sum_k c_k (1-\theta)^{a_k} \right)
\end{align}
This form is usually used for the vapor saturation density.

In data, there are three groups that are used to define the saturation pressure and densities.
\begin{lstlisting}[language=Python]
"PSgroup" : {
    "fn" : <fn index>,
    "pscale" : <p scale>,
    "Tscale" : <T scale>,
    "coef" : <polynomial coefficients>
}

"DSLgroup" : {
    "fn" : <fn index>,
    "dscale" : <d scale>,
    "Tscale" : <T scale>,
    "coef" : <polynomial coefficients>
}

"DSVgroup" : {
    "fn" : <fn index>,
    "dscale" : <d scale>,
    "Tscale" : <T scale>,
    "coef" : <polynomial coefficients>
}
\end{lstlisting}

The \texttt{fn} parameter is an index (0-3) that selects the form of the formula to use.  The \texttt{pscale} and \texttt{dscale} parameters are used to re-scale the dimensionless property calculated by the selected formula.  The \texttt{Tscale} parameter is used to normalize temperature to calculate $\theta$.  Finally, \texttt{coef} is a polynomial coefficient list described in Section \ref{sec:num:poly1}.

\subsection{Sources implemented}
As of version 2.2.0, \PM\ implements 

\begin{table}
\centering
\begin{tabular}{|ccc|}
\hline

\hline
\end{tabular}
\end{table}
